\documentclass{article}
\usepackage{amsmath, amssymb}
\usepackage{graphicx}
\usepackage{adjustbox}
\usepackage{fancyhdr}

% Title Page
\title{\textsc{Notes for CS/Math 240, Spring 2025}}
\author{\textsc{Kiplimo Kemei}}

% Page Headers
\renewcommand{\headrulewidth}{0mm} 

\pagestyle{fancy}
\fancyhf{} 
\fancyhfoffset[L]{0cm} % No left extra length
\fancyhfoffset[R]{0cm}

% Page number
\fancyhead[LE]{\thepage} 
\fancyhead[RO]{\thepage} 

% Alternating header 
\fancyhead[CO]{{{\textsc{Kiplimo Kemei}}}}
\fancyhead[CE]{{{\textsc{Notes for CS/Math 240, Spring 2024}}}}

\cfoot{} % Clear footer

\begin{document}

\maketitle

\section{Propositions and Logical Operations}
Logic is the study of formal reasoning. A statement in logic always has a well-defined meaning.

\subsection{Applications of Logic}
\begin{itemize}
    \item In Mathematics, logic is used to prove theorems.
    \item In Computer Science, logic is used in areas such as AI and in designing digital circuits.
    \item In Medicine, logic precisely specifies the conditions under which a particular diagnosis applies.
\end{itemize}

\subsection{Elements of Logic}

\subsubsection{Propositions}
A \textbf{proposition} is a statement that must be either true or false (truth value).

\textbf{Examples:}
\begin{itemize}
    \item \textit{``There are an infinite number of prime numbers.''} \quad (Truth value: True)
    \item \textit{``17 is an even number.''} \quad (Truth value: False)
\end{itemize}

\subsubsection{Proposition Variables}
Variables like $p, q,$ and $r$ can be used to denote propositions.

\textbf{Examples:}
\begin{itemize}
    \item $p$: January has 31 days.
    \item $q$: February has 33 days.
\end{itemize}

\subsection{Logical Operations}
\subsubsection{Compound Proposition}
\hbox{It will only be true if both $p$ and $q$ are true.}
\textbf{Operator:} $\land$ (and)  \\
\textbf{Example:} $p \land q$ (January has 31 days and February has 33 days)  \\
\textbf{Truth Value:} False

\subsubsection{Disjunction Proposition}
\textbf{Operator:} $\lor$ (or)  \\
\textbf{Example:} $p \lor q$ (January has 31 days or February has 33 days)  \\
\textbf{Truth Value:} True

\subsubsection{Exclusive Or (XOR)}
\textbf{Operator:} $\oplus$  \\
\textbf{Example:} $p \oplus q$ (January has 31 days or February has 33 days, but not both)  \\
\textbf{Truth Value:} True

\subsubsection{Negation}
\textbf{Operator:} $\neg$  \\
\textbf{Example:} $\neg q$ (February has 33 days)  \\
\textbf{Truth Value:} True

\end{document}
