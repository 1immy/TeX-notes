\documentclass[12pt, letterpaper]{article}
\usepackage{amsmath}% For the equation* environment
\title{ECE 203 Notes}
\author{Kiplimo Kemei\thanks{Some material is sourced from external sources.}}
\date{Spring 2024}
\begin{document}

\maketitle
Hi there, I want to preface this booklet by saying this is NOT a comprehensive 
all-you-need-to-know note document for the class ECE 203 which,
by the way, is called \emph{Signals, Information and Computation} if you 
did not know already. All the concepts talked about here will 
be related to the material discussed in class with a couple practice 
questions thrown in for good measure. Hope you find it useful!

\newpage
\tableofcontents
\newpage

\section{Course Introduction}      
Lorem  ipsum  dolor  sit  amet,  consectetuer  adipiscing  
elit.   Etiam  lobortisfacilisis sem.  Nullam nec mi et 
neque pharetra sollicitudin.  Praesent imperdietmi nec ante. 
Donec ullamcorper, felis non sodales...

\space
\begin{math}
    E=mc^2
\end{math} is typeset in a paragraph using inline math mode---as is $E=mc^2$, and so too is \(E=mc^2\).

\section{Sinusoids}   
\section{Introduction to MATLAB}   
\section{Complex Numbers, Euler's Formula, Complex Sinusoids}   
\section{Spectrum, Multiplication of Sines, AM, Periodicity}
\section{AM and beats, FM chirps, Spectrogram Lab}
\section{Fourier Series by Inspection}
\section{Assessment 1 Review}
\section{Fourier Series by Integration, Square Wave}
\section{Music Synthesis Lab}
\section{Sampling, Aliasing, Spectrum of Sampled Signals, Amplitude Quantization}
\section{Music Synthesis 2 Lab}
\section{DFT and Computing the Spectrum of Sampled Signals}
\section{Using Sinusoids to Detect Activity in fMRI Lab}
\section{DSP Systems, Impulse Response, Linearity, Time Invariance and Causality}
\section{Assessment 2 Review}


\end{document}

\end{document}