\documentclass{article}
\usepackage{amsmath, amssymb}
\usepackage{graphicx}
\usepackage{adjustbox}
\usepackage{fancyhdr}

% Page Headers
\renewcommand{\headrulewidth}{0mm} 

\pagestyle{fancy}
\fancyhf{} 
\fancyhfoffset[L]{0cm} % No left extra length
\fancyhfoffset[R]{0cm}

% Page number
\fancyhead[LE]{\thepage} 
\fancyhead[RO]{\thepage} 

% Alternating header 
\fancyhead[CO]{{{\textsc{kiplimo kemei}}}}
\fancyhead[CE]{{{\textsc{notes for linear algebra 341, spring 2025}}}}

\cfoot{} % Clear footer
% Title Page
\thispagestyle{plain}
\begin{document}
\begin{center}
\textsc{notes for linear algebra 341, spring 2025}
\break
\textsc{Kiplimo Kemei}
\end{center}
\medskip

\begin{center}
    \textsc{{1. Introduction to Vectors in $\mathbb{R}^n$}}
\end{center}

A vector $\vec{x}$ (or n-dimensional vector) is an ordered n-tuple of numbers.

\[\vec{x} = [x_1,x_2,\dots,x_n] \ \text{(row vector)}\] 

\[\vec{x} =
\begin{bmatrix}
x_1 \\
x_2 \\
\vdots \\
x_n
\end{bmatrix}
\text{(column vector)}
\]

We can represent a vector as either a row or column vector as shown above. In general, we use either of them depending on what we are trying to do with vectors. 

\medskip
For $\mathbb{R}^n$, it can be expressed as the set:

\[
\mathbb{R}^n = \left\{ 
\begin{bmatrix}
x_1 \\
\vdots \\
x_n
\end{bmatrix}
\; : \; x_1, \dots, x_n \in \mathbb{R} \right\}
\]

For \( n = 1 \), we define the real number space as:

\[
\mathbb{R}^1 = \mathbb{R}
\]

where a vector in \( \mathbb{R}^1 \) is written as:

\[
[x_1] \to x_1
\]

For \( n = 2 \), we define the two-dimensional real vector space as:

\[
\mathbb{R}^2 = \left\{ \begin{bmatrix} x_1 \\ x_2 \end{bmatrix} \; : \; x_1, x_2 \in \mathbb{R} \right\}
\]

A vector in \( \mathbb{R}^2 \) can be visualized as an arrow originating from the point \( (0,0) \) and terminating at the point \( (x_1, x_2) \). The vector is expressed in matrix notation as:

\[
\mathbf{x} =
\begin{bmatrix} x_1 \\ x_2 \end{bmatrix}, \quad
\mathbf{y} =
\begin{bmatrix} y_1 \\ y_2 \end{bmatrix}
\]

where \( x_1, x_2, y_1, y_2 \in \mathbb{R} \).

\medskip 
Two vectors \( \mathbf{x} \) and \( \mathbf{y} \) in \( \mathbb{R}^n \) are equal if and only if each corresponding component is equal. Formally, if:

\[
\mathbf{x} =
\begin{bmatrix}
x_1 \\
\vdots \\
x_n
\end{bmatrix}, \quad
\mathbf{y} =
\begin{bmatrix}
y_1 \\
\vdots \\
y_n
\end{bmatrix}
\]

then we say that \( \mathbf{x} = \mathbf{y} \) if and only if:

\[
x_j = y_j \quad \text{for } j = 1, 2, \dots, n.
\]

\medskip
\begin{center}
    \textsc{2. Vector Spaces}
\end{center}

For this class, let we will refer to a vector space using notation $V$. 

\medskip
Let $V$ be a field (a set of numbers that define the scalars used in multiplication)

\[V^n = \{(a_1,a_2,\dots,a_n)\ |\ a_i \in V \ \text{for}\ i=1,2,\dots,n\}\]

Here we have that $V^n$ is a vector space over V. In vector spaces, there are only two operations we can perform. 


\medskip
\noindent \textbf{1) Vector Addition} 

\medskip
\emph{Commutativity:} For all \( \mathbf{v}, \mathbf{w} \in V \),
\[
\mathbf{v} + \mathbf{w} = \mathbf{w} + \mathbf{v}.
\]

\emph{Associativity:} For all \( \mathbf{u}, \mathbf{v}, \mathbf{w} \in V \),
\[
(\mathbf{u} + \mathbf{v}) + \mathbf{w} = \mathbf{u} + (\mathbf{v} + \mathbf{w}).
\]

\emph{Additive Identity (Zero Vector):} There exists a special vector \( \mathbf{0} \in V \) such that for every \( \mathbf{v} \in V \),
\[
\mathbf{v} + \mathbf{0} = \mathbf{v}.
\]

\emph{Additive Inverse:} For every \( \mathbf{v} \in V \), there exists another vector \( -\mathbf{v} \in V \) such that:
\[
\mathbf{v} + (-\mathbf{v}) = \mathbf{0}.
\]

\medskip
\noindent \textbf{2) Scalar Multiplication} 

\medskip
\emph{Multiplicative Identity:} For every \( \mathbf{v} \in V \),
\[
1 \cdot \mathbf{v} = \mathbf{v}.
\]

\emph{Associativity with Scalars:} For all real numbers \( a, b \in \mathbb{R} \) and \( \mathbf{v} \in V \),
\[
a \cdot (b \cdot \mathbf{v}) = (ab) \cdot \mathbf{v}.
\]

\emph{Distributivity Over Scalar Addition:} For all real numbers \( a, b \in \mathbb{R} \) and \( \mathbf{v} \in V \),
\[
(a + b) \cdot \mathbf{v} = a \cdot \mathbf{v} + b \cdot \mathbf{v}.
\]

\emph{Distributivity Over Vector Addition:} For all real numbers \( a \in \mathbb{R} \) and vectors \( \mathbf{v}, \mathbf{w} \in V \),
\[
a \cdot (\mathbf{v} + \mathbf{w}) = a \cdot \mathbf{v} + a \cdot \mathbf{w}.
\]

\begin{center}
    \textsc{\underline{Examples of Vector Spaces}}
\end{center}

An easy example of a vector space is $\mathbb{R}^2$

Here are other important examples of vector spaces:

\medskip
\noindent \textbf{1. The set of polynomials \( \mathbb{R}[x] \)}  

This is the set of all polynomials with real coefficients. Each element is a polynomial function, such as:
\[
f(x) = 3x^2 - 2x + 5.
\]
Vector addition corresponds to adding polynomials term by term, and scalar multiplication corresponds to multiplying every term by a real number.

\medskip
\noindent \textbf{2. The set of polynomials of degree at most \( k \), denoted \( P_k(x) \)}  

This is a subset of \( \mathbb{R}[x] \) where every polynomial has degree at most \( k \). That is:
\[
P_k(x) = \{ f \in \mathbb{R}[x] \mid \deg f \leq k \}.
\]
For example, if \( k = 2 \), the set consists of all quadratic polynomials like \( ax^2 + bx + c \).

\medskip
\noindent \textbf{3. The set of \( a \times b \) matrices, denoted \( M_{a,b} \)}  

This is the set of all matrices with \( a \) rows and \( b \) columns, where each entry is a real number. A typical matrix in \( M_{2,3} \) (2 rows, 3 columns) looks like:
\[
\begin{bmatrix}
1 & 2 & 3 \\
4 & 5 & 6
\end{bmatrix}.
\]
Vector addition corresponds to matrix addition, and scalar multiplication means multiplying every entry by a real number.

\medskip
\noindent \textbf{4. The set of \( k \)-times differentiable functions, denoted \( C^k(x) \)}  

This is the set of all functions that can be differentiated at least \( k \) times, where the \( k \)th derivative exists and is continuous. If \( k = 1 \), the functions must be at least once differentiable.

\medskip
\noindent \textbf{5. The set of infinitely differentiable functions, denoted \( C^\infty(x) \)}  

This is the set of all functions that can be differentiated infinitely many times, such as:
\[
e^x, \quad \sin x, \quad \cos x.
\]
These functions have derivatives of all orders, and their derivatives never become undefined.
\end{document}